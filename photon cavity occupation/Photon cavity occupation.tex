\documentclass[12pt,reqno]{amsart}
\usepackage{geometry} % see geometry.pdf on how to lay out the page. There's lots.
\geometry{a4paper} % or letter or a5paper or ... etc
% \geometry{landscape} % rotated page geometry
\usepackage{titling}
\setlength{\droptitle}{-3cm}
% See the ``Article customise'' template for come common customisations
\usepackage{nomencl}
\makenomenclature
\title{\vspace{-1.7cm}Derivation of cavity photon number}
\author{Matthew Beck}
\date{\vspace{-.5cm}April 30, 2016} % delete this line to display the current date
%\setlength{\droptitle}{-10em} 
%%% BEGIN DOCUMENT
\begin{document}

\maketitle
\setlength{\parindent}{0pt}

%$\omega$ $\text{P}_\text{App}$
\nomenclature{$\omega_r$}{Resonant cavity frequency}
\nomenclature{$P_\text{in}$}{Power leaking into cavity}
\nomenclature{$P_\text{out}$}{Power leaking out of cavity}
\nomenclature{$P_\text{App}$}{Applied power at input of resonator}
\nomenclature{$P_\text{cav}$}{Cavity photon power}
\nomenclature{$Q_i$}{Internal quality factor}
\nomenclature{$Q_c$}{Coupling quality factor}
\nomenclature{$Q_T$}{Total quality factor}
\nomenclature{$\kappa_i$}{Photon loss rate due to internal dissipation}
\nomenclature{$\kappa_c$}{Photon loss rate due to leakage}
\nomenclature{$\kappa_T$}{Total photon loss rate}
\printnomenclature[6cm]

\section*{Derivation}

The power leaking into a cavity is proportional to the applied power at the cavity input capacitor (inductor) mediated by the coupling strength

\begin{equation}
P_{\text{in}} = P_{\text{App}} \frac{\kappa_c}{\omega_r} = \frac{P_{\text{App}}}{Q_c}
\end{equation}

Additionally, the power leaking \textit{out} of a cavity is a linear combination of the power dissipated in the cavity and the power that leaks back out of the coupling capacitor(s).

\begin{equation}
P_{\text{out}} = P_{\text{cav}} \frac{\kappa_c + \kappa_i}{\omega_r} = \frac{P_{\text{cav}}}{Q_T}
\end{equation}

At equilibrium, the powers leaking into and out of the cavity will be equal yielding

\begin{equation}\label{relation}
P_{\text{cav}} = P_{\text{App}} \frac{Q_T}{Q_c} 
\end{equation}
\setlength{\parindent}{0pt}
In the absence of applied power, we can think of the resonator as an energy source that delivers power at a rate

\begin{equation}\label{Pcav}
P_{\text{cav}} = \bar{n}\hbar \omega \kappa_T= \frac{\bar{n} \hbar \omega^2}{Q_T}
\end{equation}
Equating equations \eqref{relation} and \eqref{Pcav} we find that

\begin{equation}
\bar{n} = P_\text{App} \times \left(\frac{Q_T^2}{Q_c \hbar \omega^2}\right)
\end{equation}

where 

\begin{equation}
Q_T = \left( \frac{1}{Q_c} + \frac{1}{Q_i} \right)^{-1}
\end{equation}









\end{document}